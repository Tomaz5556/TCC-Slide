\section{Revisão da Literatura}

\begin{frame}{Revisão da Literatura}
    \begin{itemize}
        \item Tecnologias Digitais na Educação: \vspace{0.5cm}
              \begin{itemize}
                  \item Tendência irreversível no Ensino Superior; \vspace{0.5cm}
                  \item Desafios da implementação; \vspace{0.5cm}
                  \item Solução proposta para o IFNMG - Campus Salinas. \vspace{0.5cm}
              \end{itemize}
    \end{itemize}
\end{frame}

\begin{frame}{Revisão da Literatura}
    \begin{itemize}
        \item Sistemas Web: \vspace{0.25cm}
              \begin{itemize}
                  \item Importância na gestão acadêmica; \vspace{0.25cm}
                  \item Conceito. \vspace{0.25cm}
              \end{itemize}
        \item Etapas de desenvolvimento de uma aplicação web: \vspace{0.25cm}
              \begin{itemize}
                  \item Pesquisa e planejamento; \vspace{0.25cm}
                  \item Design; \vspace{0.25cm}
                  \item Desenvolvimento; \vspace{0.25cm}
                  \item Testes. \vspace{0.25cm}
              \end{itemize}
    \end{itemize}
\end{frame}

\begin{frame}{Levantamento de Requisitos}
    \begin{itemize}
        \item Etapa essencial no desenvolvimento de software; \vspace{0.5cm}
        \item Tipos de requisitos: \vspace{0.5cm}
              \begin{itemize}
                  \item Funcionais; \vspace{0.25cm}
                  \item Não funcionais. \vspace{0.5cm}
              \end{itemize}
        \item Técnica adotada: \vspace{0.25cm}
              \begin{itemize}
                  \item Entrevistas; \vspace{0.25cm}
              \end{itemize}
    \end{itemize}
\end{frame}

\begin{frame}{Front-end}
    \begin{itemize}
        \item Interface visual e interação com o usuário; \vspace{0.25cm}
        \item Projeto de Interface/Protótipos: \vspace{0.25cm}
              \begin{itemize}
                  \item A interação com sistemas informatizados se tornou indispensável; \vspace{0.25cm}
                  \item Benefícios de garantir a qualidade na construção das interfaces: \vspace{0.25cm}
                        \begin{itemize}
                            \item Confiança; \vspace{0.25cm}
                            \item Satisfação. \vspace{0.25cm}
                        \end{itemize}
              \end{itemize}
        \item Ferramenta para Design e Prototipação: \vspace{0.25cm}
              \begin{itemize}
                  \item Figma; \vspace{0.25cm}
              \end{itemize}
    \end{itemize}
\end{frame}

\begin{frame}{Ferramentas para Desenvolvimento Front-end}
    \begin{itemize}
        \item TypeScript; \vspace{0.25cm}
        \item Bibliotecas e frameworks do lado do cliente: \vspace{0.25cm}
              \begin{itemize}
                  \item React.js; \vspace{0.25cm}
                  \item Next.js: \vspace{0.25cm}
                        \begin{itemize}
                            \item Renderização híbrida; \vspace{0.25cm}
                            \item API Routes; \vspace{0.25cm}
                            \item Otimização de desempenho. \vspace{0.25cm}
                        \end{itemize}
              \end{itemize}
        \item Tailwind CSS. \vspace{0.5cm}
    \end{itemize}
\end{frame}

\begin{frame}{Back-end}
    \begin{itemize}
        \item Parte que cuida do funcionamento interno da aplicação; \vspace{0.25cm}
        \item Banco de Dados: \vspace{0.25cm}
              \begin{itemize}
                  \item Coleção organizada de dados que permite gerenciamento eficiente de informações; \vspace{0.25cm}
                  \item Tipos mais comuns: \vspace{0.25cm}
                        \begin{itemize}
                            \item Relacionais: MySQL e PostgreSQL; \vspace{0.25cm}
                            \item Não relacionais: MongoDB e Cassandra. \vspace{0.25cm}
                        \end{itemize}
                  \item Desafios: \vspace{0.25cm}
                        \begin{itemize}
                            \item Solução adotada: Uso de planilhas como banco de dados. \vspace{0.25cm}
                        \end{itemize}
              \end{itemize}
    \end{itemize}
\end{frame}

\begin{frame}{Ferramentas para Desenvolvimento Back-end}
    \begin{itemize}
        \item Java; \vspace{0.5cm}
        \item Spring; \vspace{0.5cm}
        \item Spring Boot; \vspace{0.5cm}
        \item Google Sheets; \vspace{0.5cm}
        \item Google Sheets API; \vspace{0.5cm}
        \item Docker. \vspace{0.5cm}
    \end{itemize}
\end{frame}

\begin{frame}{Integração Front-end/Back-end}
    \begin{itemize}
        \item Comunicação entre interface e servidor; \vspace{0.5cm}
        \item Principais conceitos e práticas envolvidas: \vspace{0.5cm}
              \begin{itemize}
                  \item Requisições HTTP; \vspace{0.25cm}
                  \item API REST; \vspace{0.25cm}
                  \item Serialização de dados; \vspace{0.25cm}
                  \item Autenticação e autorização; \vspace{0.25cm}
                  \item JSON. \vspace{0.25cm}
              \end{itemize}
    \end{itemize}
\end{frame}
















