\section{Metodologia}

\begin{frame}{Metodologia}
    \begin{itemize}
        \item Etapas Concluídas: \vspace{0.5cm}
              \begin{itemize}
                  \item Levantamento de Requisitos: \vspace{0.5cm}
                        \begin{itemize}
                            \item Etapa fundamental no desenvolvimento do sistema; \vspace{0.25cm}
                            \item A técnica utilizada: Entrevista; \vspace{0.25cm}
                            \item Participação: Coordenador de Ensino Superior. \vspace{0.25cm}
                        \end{itemize}
              \end{itemize}
    \end{itemize}
\end{frame}

\begin{frame}{Metodologia}
    \begin{itemize}
        \item A condução da entrevista foi realizada conforme os seguintes passos: \vspace{0.25cm}
              \begin{itemize}
                  \item Preparação: \vspace{0.25cm}
                        \begin{itemize}
                            \item Agendamento; \vspace{0.25cm}
                            \item Objetivos. \vspace{0.25cm}
                        \end{itemize}
                  \item Entrevista; \vspace{0.25cm}
                  \item Perguntas; \vspace{0.25cm}
                  \item Documentação: \vspace{0.25cm}
                        \begin{itemize}
                            \item Validação: Organização dos horários e Problemas identificados. \vspace{0.25cm}
                        \end{itemize}
              \end{itemize}
    \end{itemize}
\end{frame}

\begin{frame}{Metodologia}
    \begin{itemize}
        \item Protótipo de Interface: \vspace{0.5cm}
              \begin{itemize}
                  \item Figma; \vspace{0.5cm}
              \end{itemize}
        \item Desenvolvimento de Interface: \vspace{0.5cm}
              \begin{itemize}
                  \item Next.js; \vspace{0.5cm}
                  \item Tailwind CSS. \vspace{0.5cm}
              \end{itemize}
    \end{itemize}
\end{frame}

\begin{frame}{Metodologia}
    \begin{itemize}
        \item Google Sheets como Banco de Dados: \vspace{0.25cm}
              \begin{itemize}
                  \item Benefícios: \vspace{0.25cm}
                        \begin{itemize}
                            \item Funciona como um banco de dados leve e eficiente; \vspace{0.25cm}
                            \item Facilita o armazenamento de dados estruturados; \vspace{0.25cm}
                            \item Estrutura comparável a um SGBD convencional. \vspace{0.25cm}
                        \end{itemize}
                  \item Processo de Atualização dos Dados: \vspace{0.25cm}
                        \begin{itemize}
                            \item O setor de ensino copia os dados mais recentes de uma planilha privada; \vspace{0.25cm}
                            \item Os dados são colados na planilha utilizada como banco de dados. \vspace{0.25cm}
                        \end{itemize}
              \end{itemize}
    \end{itemize}
\end{frame}

\begin{frame}{Metodologia}
    \begin{itemize}
        \item Google Sheets API; \vspace{0.25cm}
        \item Implementação do Back-end: Spring Boot \vspace{0.25cm}
              \begin{itemize}
                  \item Essencial para estabelecer a conexão com a API do Google Sheets; \vspace{0.25cm}
                  \item Configurada para permitir apenas operações de leitura. \vspace{0.25cm}
              \end{itemize}
        \item Integração Front-end com Back-end: \vspace{0.25cm}
              \begin{itemize}
                  \item Por meio de uma API RESTful; \vspace{0.25cm}
                  \item Alterações feitas no Google Sheets apareceram automaticamente na plataforma. \vspace{0.25cm}
              \end{itemize}
    \end{itemize}
\end{frame}

\begin{frame}{Metodologia}
    \begin{itemize}
        \item Deploy: \vspace{0.5cm}
              \begin{itemize}
                  \item Front-end: Vercel \vspace{0.5cm}
                  \item Back-end: Koyeb \vspace{0.5cm}
                        \begin{itemize}
                            \item Uso do Docker; \vspace{0.25cm}
                        \end{itemize}
              \end{itemize}
    \end{itemize}
\end{frame}

\begin{frame}{Metodologia}
    \begin{itemize}
        \item Etapas Não Concluídas: \vspace{0.25cm}
              \begin{itemize}
                  \item Avaliação da Plataforma: \vspace{0.2cm}
                        \begin{itemize}
                            \item Baseada na percepção dos membros da gestão acadêmica; \vspace{0.2cm}
                            \item Objetivo: verificar se a plataforma atende às necessidades do setor. \vspace{0.2cm}
                        \end{itemize}
                  \item Abordando questões essenciais como: \vspace{0.2cm}
                        \begin{itemize}
                            \item Usabilidade e navegação; \vspace{0.2cm}
                            \item Eficiência na exibição dos horários; \vspace{0.2cm}
                            \item Redução de dificuldades operacionais; \vspace{0.2cm}
                            \item Satisfação geral; \vspace{0.2cm}
                            \item Garantia de Disponibilidade. \vspace{0.2cm}
                        \end{itemize}
              \end{itemize}
    \end{itemize}
\end{frame}

\begin{frame}{Metodologia}
    \begin{itemize}
        \item Implementação de Melhorias e Atualizações: \vspace{0.5cm}
              \begin{itemize}
                  \item Com base em sugestões e críticas na avaliação; \vspace{0.5cm}
                  \item Serão analisadas conforme: \vspace{0.5cm}
                        \begin{itemize}
                            \item Viabilidade técnica; \vspace{0.25cm}
                            \item Relevância para os usuários. \vspace{0.25cm}
                        \end{itemize}
              \end{itemize}
    \end{itemize}
\end{frame}

\begin{frame}{Metodologia}
    \begin{itemize}
        \item Documentação: \vspace{0.5cm}
              \begin{itemize}
                  \item Para garantir manutenção e funcionamento do sistema; \vspace{0.5cm}
              \end{itemize}
    \end{itemize}
\end{frame}