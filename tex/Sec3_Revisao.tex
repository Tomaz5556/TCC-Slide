\section{Revisão da Literatura}

\begin{frame}{Revisão da Literatura}
    \begin{itemize}
        \item Tecnologias Digitais na Educação: \vspace{0.5cm}
              \begin{itemize}
                  \item Tendência irreversível no Ensino Superior; \vspace{0.5cm}
                  \item Desafios da implementação; \vspace{0.5cm}
                  \item A proposta de integrar uma plataforma web à planilha busca minimizar esses desafios. \vspace{0.5cm}
              \end{itemize}
    \end{itemize}
\end{frame}

\begin{frame}{Revisão da Literatura}
    \begin{itemize}
        \item Sistema Web: \vspace{0.25cm}
              \begin{itemize}
                  \item Conceito; \vspace{0.25cm}
                  \item Importância na gestão acadêmica. \vspace{0.25cm}
              \end{itemize}
        \item Etapas comuns de desenvolvimento de uma aplicação web: \vspace{0.25cm}
              \begin{itemize}
                  \item Especificação de software; \vspace{0.25cm}
                  \item Projeto e implementação de software; \vspace{0.25cm}
                  \item Validação de software; \vspace{0.25cm}
                  \item Evolução de software. \vspace{0.25cm}
              \end{itemize}
    \end{itemize}
\end{frame}

\begin{frame}{Levantamento de Requisitos}
    \begin{itemize}
        \item Etapa essencial no desenvolvimento de software; \vspace{0.5cm}
        \item Tipos de requisitos: \vspace{0.5cm}
              \begin{itemize}
                  \item Funcionais; \vspace{0.25cm}
                  \item Não funcionais. \vspace{0.5cm}
              \end{itemize}
        \item Um aspecto fundamental é a captura dos requisitos dos usuários. \vspace{0.25cm}
              \begin{itemize}
                  \item Entrevista; \vspace{0.25cm}
              \end{itemize}
    \end{itemize}
\end{frame}

\begin{frame}{Front-end}
    \begin{itemize}
        \item Interface visual e interação com o usuário; \vspace{0.25cm}
        \item Projeto de Interface e Protótipos: \vspace{0.25cm}
              \begin{itemize}
                  \item A interação com sistemas informatizados se tornou indispensável; \vspace{0.25cm}
                  \item Benefícios de garantir a qualidade na construção das interfaces: \vspace{0.25cm}
                        \begin{itemize}
                            \item Confiança; \vspace{0.25cm}
                            \item Satisfação. \vspace{0.25cm}
                        \end{itemize}
              \end{itemize}
        \item Ferramenta para Design e Prototipação: \vspace{0.25cm}
              \begin{itemize}
                  \item Figma; \vspace{0.25cm}
              \end{itemize}
    \end{itemize}
\end{frame}

\begin{frame}{Ferramentas para Desenvolvimento Front-end}
    \begin{itemize}
        \item TypeScript; \vspace{0.25cm}
        \item Bibliotecas e frameworks do lado do cliente: \vspace{0.25cm}
              \begin{itemize}
                  \item React.js; \vspace{0.25cm}
                  \item Next.js: \vspace{0.25cm}
                        \begin{itemize}
                            \item Renderização híbrida que combina a eficiência da SSR e CSR; \vspace{0.25cm}
                            \item Confiabilidade com suporte à SSG e ISR; \vspace{0.25cm}
                        \end{itemize}
              \end{itemize}
        \item Tailwind CSS. \vspace{0.5cm}
    \end{itemize}
\end{frame}

\begin{frame}{Back-end}
    \begin{itemize}
        \item Parte que cuida do funcionamento interno da aplicação; \vspace{0.25cm}
        \item Planilhas como Banco de Dados: \vspace{0.25cm}
              \begin{itemize}
                  \item O uso de planilhas como forma de armazenamento de dados para aplicações web; \vspace{0.25cm}
                  \item Vantagens; \vspace{0.25cm}
                  \item Limitações. \vspace{0.25cm}
                  \item Muitas soluções para gerenciamento de banco de dados exigem hospedagem paga. \vspace{0.25cm}
                  \begin{itemize}
                    \item Nesse contexto, usar planilhas como banco de dados.
                  \end{itemize}
              \end{itemize}
    \end{itemize}
\end{frame}

\begin{frame}{Ferramentas para Desenvolvimento Back-end}
    \begin{itemize}
        \item Java; \vspace{0.5cm}
        \item Spring; \vspace{0.5cm}
        \item Spring Boot; \vspace{0.5cm}
        \item Google Sheets; \vspace{0.5cm}
        \item Google Sheets API; \vspace{0.5cm}
        \item Docker. \vspace{0.5cm}
    \end{itemize}
\end{frame}

\begin{frame}{Integração Front-end/Back-end}
    \begin{itemize}
        \item Comunicação entre interface e servidor; \vspace{0.5cm}
        \item Principais conceitos e práticas envolvidas: \vspace{0.5cm}
              \begin{itemize}
                  \item Requisições HTTP; \vspace{0.25cm}
                  \item API REST; \vspace{0.25cm}
                  \item Serialização de dados; \vspace{0.25cm}
                  \item Autenticação e autorização; \vspace{0.25cm}
                  \item JSON. \vspace{0.25cm}
              \end{itemize}
    \end{itemize}
\end{frame}
















